%%%%%%%%%%%%%%%%%%%%%%%%%%
% LSHTM Med Stats MSc
% Project Template
% Instructions:
% - Compile only main.tex (this file)
% - Make formating choices here
% - Put content in seperate .tex files and input them here
% - See chapter02.tex for examples of things
%%%%%%%%%%%%%%%%%%%%%%%%%%

\documentclass[11pt,a4paper,twoside,openany]{book}
% twoside: slight warning if printed will give unequal margins to allow for binding
% can be adjusted if needed; openany can be changed to open right if you want chapters
% to start on the right

% Core Formating Packages:
\usepackage{array,       % Lets you adjust table spacing
            ragged2e,    % For some layout issues 
            graphicx,    % 
            amsmath,     % 
            geometry,    % Page layout/Margins
            setspace,    % 1.5 spacing
            fancyhdr,    % Header editing
            hyperref,    % Allows links
            %indentfirst, % Uncomment for 1st para indentation
            titlesec     % Formating of chapter titles
            }

%%%% ADDITIONAL PACKAGES: %%%%          
\usepackage{helvet}
% Arial is the recommended font for MSc projects at LSHTM (bleh)
\renewcommand{\familydefault}{\sfdefault} % Sets default font to helvetica
% Arial is recommended font, Helvetica is almost identical
% If you must use Arial can use it with fontspec pkg and
% XeLatex (bleh).

\usepackage[justification=justified,   % Justifies captions
            singlelinecheck=false,     % Multiline captions with //
            labelsep=period,           % . sep, ex: Figure 1. instead of Figure 1:
            labelfont=bf]{caption}     % Bold figure/table title

\usepackage[backend=bibtex]{biblatex}
\bibliography{references.bib} % Your Bibtex file goes here
% Depending on what program you use to compile, you may
% want to remove the [backend=bibtex]

%%%% FORMATTING: %%%%
% Header:
\pagestyle{fancy}
\fancyhead{}
\fancyhead[RO]{\footnotesize\rightmark}   % Section title on right, odd pages
\fancyhead[LE]{\footnotesize\leftmark}    % Chapter Title on left, even pages
\fancyfoot[CE,CO]{\thepage}               % Centered page # in footer
\renewcommand{\headrulewidth}{0.0pt}      % Line below header
\renewcommand{\footrulewidth}{0.0pt}      % Line above footer

% Link & Metadata:
\hypersetup{
  colorlinks = true,
  urlcolor = black,    % External Links, default: blue                
  linkcolor = black,   % Internal Links, default: red  
  citecolor= black,    % citations, deafault: lime green  (?)
  pdfauthor = {},      % Don't put your name or it will show up in metadata
  pdfkeywords = {},
  pdftitle = {MSc Project Report},
  pdfsubject = {},
  pdfpagemode = UseNone
}
 
% Path for Images:
\graphicspath{{figures/}}  

% Margins:
\geometry{
  left=1in, right=1in,
  top=1in, bottom=1in,
  head=0.5in, foot=0.5in
}

% Chapter Title Formating w/ titlesec pkg:
\titleformat{\chapter}{\normalfont\huge}{\thechapter.}{20pt}{\huge\bf\scshape}
% scshape: Big/Small Caps
% Removes "Chapter X" titles in header & before text:
\renewcommand{\chaptername}{}

% Prevents latex from vertically trying to 
% fill pages with ragged2e command:
\raggedbottom

%%%% Other options you may want %%%%
% Used to change table horzontal spacing:
% \renewcommand{\arraystretch}{1.0}
% Table vertical spacing:
 {\setlength{\extrarowheight}{1pt}
% Used to change default indent spacing:
% \setlength{\parindent}{2em}
% Spacing between paragraphs:
% \setlength{\parskip}{0em}

%%%%%%%%%%%%%%%%%%%%%%%%%%%
% DOCUMENT
%%%%%%%%%%%%%%%%%%%%%%%%%%%

\begin{document}

%%%% Front Matter %%%%
\frontmatter

\begin{titlepage}

  \begin{flushright}
   \includegraphics[width=0.3\textwidth]{schoolseal.jpg}
  \end{flushright}
  
  \begin{center}
    \LARGE
        MSc Project Report \\
        2017--2018
    \vspace*{1cm}
        
    \Huge
        \textbf{Title of the Report}
    \vspace{2.5cm}
        
    \LARGE
        \textbf{Candidate Number:} 12345678 \\
        \textbf{Supervisor:} Dr. Supervisor \\
    \vspace{0.5cm}
    \large
        Page Count:
    \vfill
    \Large
        Submitted in part fulfilment of the requirements for \\
        the degree of MSc in Medical Statistics \\
    \vspace{0.8cm}
    \large
        September 2017 \\
        
  \end{center}
  
\end{titlepage}

  % Title Page

\onehalfspacing         % 1.5 Spacing, for double: \doublespacing
  \begin{center}
\textbf{Acknowledgements}
\end{center}
\addcontentsline{toc}{section}{Acknowledgements}

Blah Blah
\newpage                % Seperate ack & abstract
  \begin{center}
\textbf{Abstract}
\end{center}
\addcontentsline{toc}{section}{Abstract}

Blah Blah

\tableofcontents
\listoffigures
\listoftables

%%%% Main Matter %%%%

\mainmatter
  \chapter{Introduction}

\section{Introduction Section 1}

\subsection{Introduction Subsection 1}
This is text for your introduction.
\subsection{Introduction Subsection 2}
\subsection{Introduction Subsection 3}


\section{Introduction Section 2}

\subsection{Introduction Subsection 1}
\subsection{Introduction Subsection 2}
\subsection{Introduction Subsection 3}

  \input{sections/chapter02}
  \input{sections/chapter03}
  \input{sections/chapter04}
  \input{sections/conclusion}

%%%% Back Matter %%%%

\backmatter
\printbibliography[title = {References}]
\addcontentsline{toc}{chapter}{References}
\appendix
\input{sections/appendix}

\end{document}